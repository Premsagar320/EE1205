\let\negmedspace\undefined
\let\negthickspace\undefined
\documentclass[journal,12pt,twocolumn]{IEEEtran}
\usepackage{cite}
\usepackage{amsmath,amssymb,amsfonts,amsthm}
\usepackage{algorithmic}
\usepackage{graphicx}
\usepackage{textcomp}
\usepackage{xcolor}
\usepackage{txfonts}
\usepackage{listings}
\usepackage{enumitem}
\usepackage{mathtools}
\usepackage{gensymb}
\usepackage{comment}
\usepackage[breaklinks=true]{hyperref}
\usepackage{tkz-euclide} 
\usepackage{listings}
\usepackage{gvv}                                        
\def\inputGnumericTable{}                                 
\usepackage[latin1]{inputenc}                                
\usepackage{color}                                            
\usepackage{array}                                            
\usepackage{longtable}                                       
\usepackage{calc}                                             
\usepackage{multirow}                                         
\usepackage{hhline}                                           
\usepackage{ifthen}                                           
\usepackage{lscape}

\newtheorem{theorem}{Theorem}[section]
\newtheorem{problem}{Problem}
\newtheorem{proposition}{Proposition}[section]
\newtheorem{lemma}{Lemma}[section]
\newtheorem{corollary}[theorem]{Corollary}
\newtheorem{example}{Example}[section]
\newtheorem{definition}[problem]{Definition}
\newcommand{\BEQA}{\begin{eqnarray}}
\newcommand{\EEQA}{\end{eqnarray}}
\newcommand{\define}{\stackrel{\triangle}{=}}
\theoremstyle{remark}
\newtheorem{rem}{Remark}
\begin{document}

\bibliographystyle{IEEEtran}
\vspace{3cm}

\title{GATE 2022 IN 14}
\author{EE23BTECH11065 - prem sagar}
\maketitle
\newpage

\bigskip

\renewcommand{\thefigure}{\theenumi}
\renewcommand{\thetable}{\theenumi}
\textbf{Question}:
\\The output of the system y\brak{t} is related to its input x\brak{t} according to the relation $y\brak{t}=x\brak{t}sin\brak{2\pi t}$.This system is 
\\\\\brak{A} Linear and time-variant
\\\brak{B} Non-Linear and time-invariant
\\\brak{C} Linear and time-invariant
\\\brak{D} Non-linear and time-variant
\\\textbf{Solution}:
\begin{table}[!ht]
\def\arraystretch{1.5}
   \centering
    \renewcommand\thetable{1}
      \begin{tabular}{|c|c|c|}
   \hline
   \textbf{Symbol} & \textbf{Value}& \textbf{Description} \\
   \hline
         $x\brak{t}$ & & input signal\\
        \hline
        $y\brak{t}$ &$x\brak{t}sin\brak{2\pi t}$  & output signal\\
        \hline
        $\tau$ &   & Time delay\\
        \hline
\end{tabular}

    \caption{input parameters}
    \label{tab:IN 14}
 \end{table}
\\ From \tabref{tab:IN 14}
\\\begin{align}
y_1\brak{t}&\leftrightarrow x_1\brak{t}
\\y_2\brak{t}&\leftrightarrow x_2\brak{t}
\\ay_1\brak{t}+by_2\brak{t}&\leftrightarrow ax_1\brak{t}+bx_2\brak{t}
\\ay_1\brak{t}+by_2\brak{t}&=\brak{ax_1\brak{t}+bx_2\brak{t}}sin\brak{2\pi t}
\end{align}
\\$\therefore$ satisfies principle of superposition
\begin{align}
ky\brak{t}&\leftrightarrow kx\brak{t}
\\ky\brak{t}&=k\brak{x\brak{t}sin\brak{2\pi t}}
\end{align}
\\$\therefore$ satisfies principle of homogenity
\\$\therefore$ it is linear
\\\\Delay in input x\brak{t}:
\begin{align}
y_1\brak{t}&=x\brak{t-\tau}sin\brak{2\pi t}
\end{align}
\\\\Delay in output y\brak{t}:
\begin{align}
y\brak{t-\tau}&=x\brak{t-\tau}sin\brak{2\pi\brak{t-\tau}}
\\y_2\brak{t}&=x\brak{t-\tau}sin\brak{2\pi\brak{t-\tau}}
\\y_1\brak{t}&\neq y_2\brak{t}
\end{align}
\\$\therefore$ it is time variant
\\$\therefore$ \brak{A} linear and time variant 
\end{document}

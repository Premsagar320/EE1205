\let\negmedspace\undefined
\let\negthickspace\undefined
\documentclass[journal,12pt,twocolumn]{IEEEtran}
\usepackage{cite}
\usepackage{amsmath,amssymb,amsfonts,amsthm}
\usepackage{algorithmic}
\usepackage{graphicx}
\usepackage{textcomp}
\usepackage{xcolor}
\usepackage{txfonts}
\usepackage{listings}
\usepackage{enumitem}
\usepackage{mathtools}
\usepackage{gensymb}
\usepackage{comment}
\usepackage[breaklinks=true]{hyperref}
\usepackage{tkz-euclide} 
\usepackage{listings}
\usepackage{gvv}                                        
\def\inputGnumericTable{}                                 
\usepackage[latin1]{inputenc}                                
\usepackage{color}                                            
\usepackage{array}                                            
\usepackage{longtable}                                       
\usepackage{calc}                                             
\usepackage{multirow}                                         
\usepackage{hhline}                                           
\usepackage{ifthen}                                           
\usepackage{lscape}

\newtheorem{theorem}{Theorem}[section]
\newtheorem{problem}{Problem}
\newtheorem{proposition}{Proposition}[section]
\newtheorem{lemma}{Lemma}[section]
\newtheorem{corollary}[theorem]{Corollary}
\newtheorem{example}{Example}[section]
\newtheorem{definition}[problem]{Definition}
\newcommand{\BEQA}{\begin{eqnarray}}
\newcommand{\EEQA}{\end{eqnarray}}
\newcommand{\define}{\stackrel{\triangle}{=}}
\theoremstyle{remark}
\newtheorem{rem}{Remark}
\begin{document}

\bibliographystyle{IEEEtran}
\vspace{3cm}

\title{11.9.3.3}
\author{EE23BTECH11065 - prem sagar}
\maketitle
\newpage

\bigskip 

\renewcommand{\thefigure}{\theenumi}
\renewcommand{\thetable}{\theenumi}
\textbf{Question}:\\ The 5th,8th and 11th terms of a GP are p,q and s respectively .show that \[q^2=ps\]
\textbf{solution}:
\\\begin{table}[!ht]
   \centering
    \renewcommand\thetable{1}
        \begin{tabular}{|c|c|c|}
    \hline
            \textbf{Symbol} & \textbf{Value} & \textbf{Description} \\
    \hline
          $x(5)$ & $p$ & 5th term of G.P \\
    \hline
          $x(8)$ & $q$ & 8th term of G.P\\
    \hline 
          $x({11})$ &$s$ &11th term of G.P \\
    \hline
  \end{tabular}

    \caption{input parameters}
    \label{tab:table_11.9.3}
 \end{table}
\\ From \tabref{tab:table_11.9.3}:
\begin{align}
\\x\brak{8}\,x\brak{8}&=x\brak{0}\,r^8\,x\brak{0}\,r^8
     \\ &=x\brak{0}^2\,r^{16}
\\x\brak{5}\,x\brak{{11}}&=x\brak{0}\,r^5\,x(0)\,r^{11}
       \\&=x\brak{0}^2\,r^{16}
\\x\brak{8}^2&=x\brak{5}\,x\brak{{11}}
\\q^2&=ps
\end{align}   
\\\begin{figure}[h]
   \renewcommand\thefigure{1}
    \centering
    \includegraphics[width=1\linewidth]{/root/assign1/figs/figure__plot.png}
    \caption{plot x\brak{n}vs n\hspace{0.1cm}p=486,
    \hspace{0.1cm}q=13122,
    \hspace{0.1cm}s=118098}
    \label{fig:1}
\end{figure}\\
\\\begin{align}
x\brak{n}\overset{Z}{\longleftrightarrow}   X\brak{Z}
\\x\brak{n}&=x\brak{0}\,r^n\,u\brak{n}
\\X\brak{Z}&=\sum_{n=-\infty}^{\infty}x\brak{n} Z^{-n}\
     \\ &= \frac{x\brak{0}}{1-r\,z^{-1}}\: \:,|z|>|r|
     \end{align}
\end{document}

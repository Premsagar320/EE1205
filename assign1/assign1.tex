\let\negmedspace\undefined
\let\negthickspace\undefined
\documentclass[journal,12pt,twocolumn]{IEEEtran}
\usepackage{cite}
\usepackage{amsmath,amssymb,amsfonts,amsthm}
\usepackage{algorithmic}
\usepackage{graphicx}
\usepackage{textcomp}
\usepackage{xcolor}
\usepackage{txfonts}
\usepackage{listings}
\usepackage{enumitem}
\usepackage{mathtools}
\usepackage{gensymb}
\usepackage{comment}
\usepackage[breaklinks=true]{hyperref}
\usepackage{tkz-euclide} 
\usepackage{listings}
\usepackage{gvv}                                        
\def\inputGnumericTable{}                                 
\usepackage[latin1]{inputenc}                                
\usepackage{color}                                            
\usepackage{array}                                            
\usepackage{longtable}                                       
\usepackage{calc}                                             
\usepackage{multirow}                                         
\usepackage{hhline}                                           
\usepackage{ifthen}                                           
\usepackage{lscape}

\newtheorem{theorem}{Theorem}[section]
\newtheorem{problem}{Problem}
\newtheorem{proposition}{Proposition}[section]
\newtheorem{lemma}{Lemma}[section]
\newtheorem{corollary}[theorem]{Corollary}
\newtheorem{example}{Example}[section]
\newtheorem{definition}[problem]{Definition}
\newcommand{\BEQA}{\begin{eqnarray}}
\newcommand{\EEQA}{\end{eqnarray}}
\newcommand{\define}{\stackrel{\triangle}{=}}
\theoremstyle{remark}
\newtheorem{rem}{Remark}
\begin{document}

\bibliographystyle{IEEEtran}
\vspace{3cm}

\title{11.9.3.3}
\author{EE23BTECH11065 - prem sagar}
\maketitle
\newpage

\bigskip

\renewcommand{\thefigure}{\theenumi}
\renewcommand{\thetable}{\theenumi}
\textbf{Question}:\\ The 5th,8th and 11th terms of a GP are p,q and s respectively .show that \[q^2=ps\]
\textbf{solution}:
\\Given,
\\$a_5=p$
\\$a_8=q$
\\$a_{11}=s$
\\let first term of a GP= a\\
common ratio of GP=r
\\we know,
\\ nth term of  a $GP(a_n)$$= a\cdot r^{n-1}$ 
\\so 5th  term of $GP(a_5)$$= a\cdot r^4=p$
\\8th  term  of  $GP(a_8)$$= a\cdot r^7=q$
\\11th  term  of  $GP(a_{11})$$= a\cdot r^{10}=s$
\\$a_8\cdot a_8=a\cdot r^7\cdot a\cdot r^7$
    \\$=a^2\cdot r^{14}$
\\$a_5\cdot a_{11}=a\cdot r^4\cdot a\cdot r^{10}$
     \\ $=a^2\cdot r^{14}$
\\$a_8^2=a_5\cdot a_{11}$
\\therefore,
\\$q^2=p\cdot s$
\\hence proved
\\$q^2=p\cdot s$
\\$\\\begin{tabular}{|c|c| }
\hline
\textbf{n}& \textbf{value}
\\\hline
\multirow{3}{1em}\\1 &$a$
\\2 &$a\cdot r$
 \\3 &$a\cdot r^2$ 
 \\4 &$a\cdot r^3$ 
 \\5 &$a\cdot r^4$
 \\6 &$a\cdot r^5$
 \\7 &$a\cdot r^6$
 \\8 &$a\cdot r^7$
 \\9 &$a\cdot r^8$
 \\{10} &$a\cdot r^9$
 \\{11} &$a\cdot r^{10}$
  \\\hline
\end{tabular}$\\
\\$\\\begin{tabular}{|c|c| }
\hline
\textbf{term}& \textbf{value}
\\\hline
\multirow{3}{1em}\\$a_5$ &$p$
\\$a_8$ &$q$
\\$a_{11}$ &$s$
  \\\hline
\end{tabular}$\\
\\so,
\\$p=a\cdot r^4$
\\$q=a\cdot r^7$
\\$s=a\cdot r^{10}$
\\hence proved
\\$q^2=p\cdot s$
\end{document}

